\section{Requirements Engineering}
The following will contain a list of concrete requirements for the project.
The list is not final for the moment as future updates and development
processes may impact the requirements of the project.

The section \ref{sec:use cases} will contain a list of use cases which are the
base for all requirements. The requirements themselves will be listed in
section \ref{sec:func_requirements}.

\subsection{Use Cases}\label{sec:use cases}
The following contains a list of use cases for the system. 

All use cases result in the construction of a PDF-Document which contains full
images and is free from length marks.

\subsubsection*{Removing the draft-option}
The user calls the pipetex-application by entering the CLI-command for the
application and supplying the name of the main {\LaTeX} file.  This is done in
the directory containing the {\LaTeX} file.

The application will make a copy or the file, remove draft option from the
class definition and compile the {\LaTeX} file two times. The compiled pdf file
will be moved to a separate directory and all generated auxiliary files will be
removed from the directory.

\subsubsection*{Creating a bibliography}
The user calls the pipetex-application by entering the CLI-command for the
application and supplying the name of the main {\LaTeX} file and a flag which
specifies the creation of a bibliography file.  This is done in the directory
containing the {\LaTeX} file.

The application will make a copy or the file, remove draft option from the
class definition and compile the {\LaTeX} file. The application will call a
\gls{bib_engine} on the {\LaTeX} file and then compile the file again. The
compiled pdf file will be moved to a separate directory and all generated
auxiliary files will be removed from the directory.

\subsubsection*{Creating a glossary}
The user calls the pipetex-application by entering the CLI-command for the
application and supplying the name of the main {\LaTeX} file and a flag which
specifies the creation of a glossary file.  This is done in the directory
containing the {\LaTeX} file.

The application will make a copy or the file, remove draft option from the
class definition and compile the {\LaTeX} file. The application will call a
\gls{glo_engine} on the {\LaTeX} file and then compile the file again. The
compiled pdf file will be moved to a separate directory and all generated
auxiliary files will be removed from the directory.

\subsection{Functional Requirements}\label{sec:func_requirements}
Now following is a compilation of all requirements for the application.  The
requirements will be grouped into three categories.  The categories are ordered
by importance:

\begin{itemize}
    \item \Gls{must_have_req}
    \item \Gls{should_have_req}
    \item \Gls{nice_to_have_req}
\end{itemize}

The requirements have an ID, a name and a short description.

\newpage
\subsubsection{Must Have Requirements}\label{sec:must_have_requirements}
The following contains all \Gls{must_have_req}. 

\begin{table}[!ht]
    \centering
    \begin{tabular}{ | p{0.15\textwidth} | p{0.3\textwidth} | p{0.45\textwidth} | }
        \hline
        \textbf{ID} & \textbf{Name} & \textbf{Description} \\
        \hline
        MHR1 & {\LaTeX}-Compilation & The application must be able to compile {\LaTeX} Documents using a {\TeX}-Engine. \\ \hline
        MHR2 & Draft-Mode-Removal & The application must be able to find and remove the draft option in the class definition \\ \hline
        MHR3 & Creation of bibliography &  The application must be able to run a script to create a bibliography on the document \\ \hline
        MHR4 & Creation of glossary & The application must be able to run a script to create a glossary on the document \\ \hline
        MHR5 & Working-dir-agnosticism & The script must be callable from everywhere in the file system without the need of a copy in any specific location \\ \hline
        MHR7 & CLI-Compatibility & The application must be callable from a CLI \\ \hline
        MHR8 & Lossless-Execution & The application must not make changes to the main {\TeX} file. Instead, it should operate on a copy of the file to ensure nothing is lost. \\ \hline
        MHR9 & Working-Dir-Cleanup& The application should remove all generated auxiliary files once the compilation process is over. \\ \hline
    \end{tabular}
    \caption{Functional Must Have Requirements}
\end{table}

\newpage
\subsubsection{Should Have Requirements}\label{sec:should_have_requirements}
The following contains all \Gls{should_have_req}. 

\begin{table}[!ht]
    \centering
    \begin{tabular}{ | p{0.15\textwidth} | p{0.3\textwidth} | p{0.45\textwidth} | }
        \hline
        \textbf{ID} & \textbf{Name} & \textbf{Description} \\
        \hline
        SHR1 & Meaningful-Exit-on-Error & The application should only quit if the {\LaTeX}-compilation fails. All other engines should log the error and continue with the execution \\ \hline
        SHR2 & Logging & The application should output log messages to the CLI. \\ \hline
        SHR3 & Changing-Output-Name & The application should provide the option to rename the PDF document which will be generated. \\ \hline
    \end{tabular}
    \caption{Functional Should Have Requirements}
\end{table}

\subsubsection{Nice To Have Requirements}\label{sec:nice_to_have_requirements}
The following contains all \Gls{nice_to_have_req}. 

\begin{table}[!ht]
    \centering
    \begin{tabular}{ | p{0.15\textwidth} | p{0.3\textwidth} | p{0.45\textwidth} | }
        \hline
        \textbf{ID} & \textbf{Name} & \textbf{Description} \\
        \hline
        NtHR1 & Colored-Logs & The logs printed to the CLI should be colored. \\ \hline
        NtHR2 & Log-File & The application should provide the option to redirect the logs to a logfile. \\ \hline
        NtHR3 & Verbose-Mode & The application should provide the option to print the output of the {\TeX} / Bib / Gloss engine to the CLI. \\ \hline
        NtHR4 & Changing-Logging-Level & The application should provide an option to increase or decrease the logging level. \\ \hline
        NtHR5 & Changing-Engine & The application should provide a way to change specific compilation engines if the user is not satisfied with the standard. \\ \hline
    \end{tabular}
    \caption{Functional Nice to Have Requirements}
\end{table}


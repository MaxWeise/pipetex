\section{Styleguide}

\subsection{General Guides}
As I'm sure the list following in section~\ref{sub:decisions} will be
incomplete for some time to come, as some situations won't occur or are not
foreseen at the time of writing. As it would be too much effort to address every
edge case I present the following compromise:

Most situations handling codestyle problems are handled in
section~\ref{sub:descitions}. If for some reason a situation is not handled
there, please refer to the Google Python Style Guide. 
% \url{https://google.github.io/styleguide/pyguide.html}
In the rare case that something is not handled \emph{there}, please refer to the
official PEP8 Style Guide written by Guido van Rossum.
Please note, that neither the style guide provided by Google, not the PEP8 Guide
overrule the decisions made in this style guide.

\subsection{Tools}
Following will list the tools used to help in keeping the code clean
and uniform:

\subsubsection{Linter}
Currently, the \textbf{flake8} linter is used. The repository also provides a
flake8rc which configures the linter according to the requirements. To run the
linter, simply run it on the source directory using \cli{flake8 src/}. I also
recommend integrating it into your text editor or IDE to run it automatically
on the edited file or on the press of a button. 

\subsubsection{Typechecking}
As typehints are greatly encouraged in the project, I recomend to
use \textbf{mypy} when modifying the code. In this case, no configuration is
needed and you can simply run \cli{mypy src/} on the source directory to check
for any type incompatibilities. This can also be integrated into your
text editor or IDE.

\subsection{Style Decisions}\label{sub:descitions}
Now, on to how the code actually should look.

\subsubsection{Project Layout}
The rogh layout of the project looks like this: 

\begin{figure}[!ht]
    % \centering
    \dirtree{%
    .1 /.
        .2 docs.
        .2 src.
            .3 pipetex.
                .4 \_\_init\_\_.py.
                .4 main.py.
                .4 pipeline.py.
                .4 various\_utils.py\DTcomment{This is just an example module}.
        .2 tests.
    }
    \caption{Directory Structure}
\end{figure}

Ideally, only new modules have to be introduced to the codebase and no new
packages.

\newpage

\subsubsection{Modules}
Modules should be named with descriptive, snake\_case names. In the best case
scenario, one word modules should be enough, but multiple words are possible as
a module name.

Every module should start with a docstring which is used to generate
documentation.  The docstrings should include \emph{what} belongs inside the
module, why is it needed and (if possible) a common use case description.
Please also include the name of the author and the date of creation. For more
information on how to write docstrings (not only for modules) please refer to
section~\ref{ssub:docstrings}. A module docstring may look something like this:


\lstinputlisting[
    language=Python,
    caption={Example Module Docstring},
    firstline=1,
    lastline=12,
    columns=flexible,
]
{../src/pipetex/operations.py}


Modules may contain multiple classes, especially if they are small in lines
of code and have some semantic or logical connection to each other (e.g:
enumerations can be located in the same module). As classes are getting bigger
or need to be more isolated for testing purposes they should be moved to a
separate file.

\subsubsection{Classes}
Classes are a great feature for combining data and behavior, implementing
design patterns and encapsulating similar things together. Python offers the
typical behavior of a object oriented programming language and it is usually a
good idea to harvest the benefits of this. But luckily, we are not using Java
(no offense) and are not \emph{forced} to rely on classes. So before
implementing, take a moment to reflect on the planned implementation and
determine if a class is the best option for the task or if it can be realized
by for example a simple function.

I usually use 3 questions when determining if I need a class or not.

\begin{itemize}
   \item Do I need to store data in the class? 
   \item Do I need to reuse the class?
   \item Do I need the class for a type definition or inheritance hierarchy?
\end{itemize}

If you answer 'yes' to at least two of these questions then its probably best
to use a class. Otherwise consider using functions, especially if you don't
need to store data in an object.

Classes should be named in the PascalCase style, numbers and symbols like underscores
or dashes are discouraged. Each class must contain a docstring below the class
definition describing the class and listing any public attributes.


\subsubsection{Docstrings}\label{ssub:docstrings}



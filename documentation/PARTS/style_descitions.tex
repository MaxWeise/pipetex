\section{Styleguide}

\subsection{General Guides}
As I'm sure the list following in section~\ref{sub:descitions} will be
incomplete for some time to come, as some situations won't occur or are not
forseen at the time of writing. As it would be too much effort to adress every
edge case I present the following compromise:

Most situations handling codestyle problems are handled in
section~\ref{sub:descitions}. If for some reason a situation is not handled
there, please refer to the Google Python Style Guide. 
% \url{https://google.github.io/styleguide/pyguide.html}
In the rare case that something is not handled \emph{there}, please refer to the
official PEP8 Style Guide written by Guido van Rossum.
Please note, that neither the style guide provided by Google, not the PEP8 Guide
overrule the descitions made in this style guide.

\subsection{Tools}
Following will list the tools used to help in keeping the code clean
and uniform:

\subsubsection{Linter}
Currently, the \textbf{flake8} linter is used. The repository also provides a
flake8rc which configures the linter according to the requirements. To run the
linter, simply run it on the source directoy using \cli{flake8 src/}. I also
recomend integrating it into your texteditor or IDE to run it automatically
on the edited file or on the press of a button. 

As typehints are greatly encouraged in the project, I recomend to
use \textbf{mypy} when modifying the code. In this case, no configuration is
needed and you can simply run \cli{mypy src/} on the source directoy to check
for any type incompatabilities. This can also be integrated into your
texteditor or IDE.

\subsection{Style Descitions}\label{sub:descitions}
<++>


\section{Introduction}
Hi, welcome to this requirements document. It is a collection of
key-characteristics and features this project should include. The whole purpose
of this project is to learn and improve my coding and software development
skills. But let's not get ahead of ourselves.

In section \ref{sec:motivation} I will briefly talk about why this application
is useful to me and might potentially be useful to you and section
\ref{sec:goals} will set the goals which determine the success of this project.

So without further ado, let's jump into it.


\subsection{Motivation}\label{sec:motivation}
The motivation for this project is twofold:

\begin{enumerate}
    \item Keep expanding my knowledge in the software development field
    \item Improving my workflow with {\LaTeX}-documents
\end{enumerate}

Let's talk about them separately for a bit.

\subsubsection*{Gaining Knowledge}
First and foremost, I wanted to create a more or less complete application from
start to {\glqq finish\grqq}, formulating requirements, designing software on
the drawing board and packaging everything up to a nice installable package for
me (and maybe others) to use.

I had the chance to participate in one or two development projects as part of
my studies and I definitely learned much from them. The downside was, that none
of them was a \emph{complete} development process. Complete in the sense of
including all vital steps. In the case of a project which included a huge
codebase we were included in the SCRUM process and we would have regular
meetings where we presented our work and which tickets were next to be worked
on. The development part was very spot-on, but the planning and designing part
was pretty much nonexistent.

Other projects were simple projects to be submitted at the end of the semester.
While the expectations of my professors were higher each year and we had to be
more specific about what we wanted to do, the project never actually came to a
packaging and deployment phase.

So in essence, I have kind of experienced a whole development process, but
split over multiple projects and semesters. So this project should combine the
experience from all my previous projects and let me face each challenge on my
own and without the pressure of good grades lurking somewhere.

\subsubsection*{Improving Workflow}
{\LaTeX} is a great tool to create PDF documents which look professional
and uniform. I use it pretty much for every mildly important documents and
even this requirements specification file is written and compiled using latex.

While it is a great tool, there are some inconveniences I'd like to get rid of.
One of those inconveniences is the so called {\glqq draft mode\grqq}. This is
mainly used to minimize compilation time by only showing black frames instead
of fully rendered images. While it is nice in the early stages of the document
because it speeds up the workflow, it becomes tedious when the final document
has to be created over and over again, but the draft option can't be removed 
fully, so I have to manually remove and re-add it every time I want to switch
modes. Sometimes I have to check if the full image is big enough so I can't
use draft mode for this. And in some cases (like this document) I keep it in
draft mode all the time and only switch to final mode when I need to deploy the
document.

And don't mention bibliographies or glossaries. Each of them requires a
separate engine to be run on the document, sometimes multiple times and in a
very specific order, otherwise the references won't work or won't even be
visible. And don't let me mention the various auxiliary files generated by all
the engines and features {\LaTeX} provides.

Wouldn't it be neat if only I can run a single command and all those tasks will
be done for me? Of course I could configure my texteditor to do all those
things for me using commands and keyboard-shortcuts and whatnot, but even those
I'd have to input manually and I'd have to wait until the first step is
finished, then input the second instruction, wait for that to finish executing
and so on.


\subsection{Goals}\label{sec:goals}
Now, lets define the goals for the project. Some of the {\glqq goals\grqq} I
listed above are really more \emph{intentions}, as they aren't formulated sharp
enough to be objectively measured or it's just their nature to be vague and
abstract.

Keep in mind the bullet points below are not actually the requirements for the
project, these will be listed further down (however, requirements can serve as
goals in my opinion).

I envision a system which will be fed some basic information, like which tasks
to run and the file which will be compiled and then do them automatically, one
after the other and without me having to babysit the computer during the
process and telling it when to do what. It might take a bit longer, but even if
the time is long enough for me to grab a coffee, the system will be worth it
for my purposes, as I work very frequently in {\LaTeX}.

Following is a list of bulletpoints which I want to accomplish with this
application.

\begin{itemize}
    \item Building a package in Python which builds {\LaTeX} documents and
        cleans up the working environment. This includes:
        \begin{itemize}
            \item Running a {\LaTeX}-engine which compiles to PDF-documents
            \item Running a bibliography-engine like biber or bibtex.
            \item Running a glossary engine like makeglossaries
        \end{itemize} 
    \item Make the package installable
    \item Make the script be callable from everywhere in the directory
        structure.  This means that the script should be run an function no
        matter where it is called and / or installed.
\end{itemize}


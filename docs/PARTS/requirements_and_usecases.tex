\section{Requirements Engineering}
The following will contain a list of concrete requirements for the project.
The list is not final for the moment as future updates and development
processes may impact the requirements of the project.

The section \ref{sec:use cases} will contain a list of use cases which are the
base for all requirements. The requirements themselves will be listed in
section \ref{sec:requirements}.

\subsection{Use Cases}\label{sec:use cases}
The following contains a list of use cases for the system. 

All use cases result in the construction of a PDF-Document which contains full
images and is free from length marks.

\subsubsection*{Removing the draft-option}
The user calls the pipetex-application by entering the CLI-command for the
application and supplying the name of the main {\LaTeX} file.  This is done in
the directory containing the {\LaTeX} file.

The application will make a copy or the file, remove draft option from the
class definition and compile the {\LaTeX} file two times. The compiled pdf file
will be moved to a separate directory and all generated auxiliary files will be
removed from the directory.

\subsubsection*{Creating a bibliography}
The user calls the pipetex-application by entering the CLI-command for the
application and supplying the name of the main {\LaTeX} file and a flag which
specifies the creation of a bibliography file.  This is done in the directory
containing the {\LaTeX} file.

The application will make a copy or the file, remove draft option from the
class definition and compile the {\LaTeX} file. The application will call a
bibliography engine on the {\LaTeX} file and then compile the file again. The
compiled pdf file will be moved to a separate directory and all generated
auxiliary files will be removed from the directory.

\subsubsection*{Creating a glossary}
The user calls the pipetex-application by entering the CLI-command for the
application and supplying the name of the main {\LaTeX} file and a flag which
specifies the creation of a glossary file.  This is done in the directory
containing the {\LaTeX} file.

The application will make a copy or the file, remove draft option from the
class definition and compile the {\LaTeX} file. The application will call a
glossary engine on the {\LaTeX} file and then compile the file again. The
compiled pdf file will be moved to a separate directory and all generated
auxiliary files will be removed from the directory.

\subsection{Requirements}\label{sec:requirements}
Now following is a compilation of all requirements for the application.  The
requirements will be grouped into three categories.  The categories are ordered
by importance:

\begin{itemize}
    \item Must Have Requirements
    \item Should Have Requirements
    \item Nice To Have Requirements
\end{itemize}

The requirements have an ID, a name and a short description.

% TODO: Add table containing MH Requirements, SH Requirements and NtH Requirements

\subsubsection{Must Have Requirements}\label{sec:must_have_requirements}
\subsubsection{Should Have Requirements}\label{sec:should_have_requirements}
\subsubsection{Nice To Have Requirements}\label{sec:nice_to_have_requirements}
